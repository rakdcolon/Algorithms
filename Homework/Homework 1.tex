\documentclass[boxes]{rutgers_hw}
\usepackage{rutgers}
\usepackage{multicol}
\newcommand{\Mod}[1]{\ (\mathrm{mod}\ #1)}
% \usepackage[none]{hyphenat} % Use to avoid hyphens



\author{Rohan Karamel} % Enter your name
\netid{rak218} % Enter your NetID or comment out
% \collaborators{Leonhard Euler, Bernard Bolzano} % Enter your collaborators or comment out
\assignment{Homework 1} % Enter the assignment name
\date{\today} % Replace with due date
\course{Algorithms} % Enter the course name
\semester{Spring 2024} % Enter the semester
\sectionnum{2} % Enter your section number
\instructor{Professor Szegedy} % Enter your professor's name
\institution{Rutgers University} % Enter your university
\renewcommand{\labelenumi}{\alph{enumi}}
\begin{document}

\maketitle

% You can comment out code by using the percentage symbol


\begin{exern}{1} % This is a manually numbered exercise.
  Book 0.1
\end{exern}
\begin{solution} % This is a solution environment. 
  \hfill
  \begin{enumerate} 
    \begin{multicols}{2}
    
      \item $f = \Theta(g)$
      \item $f = O(g)$
      \item $f = \Theta(g)$
      \item $f = \Theta(g)$
      \item $f = \Theta(g)$
      \item $f = \Theta(g)$
      \item $f = O(g)$
      \item $f = \Omega(g)$
      \item $f = \Omega(g)$
      \item $f = \Omega(g)$
      \item $f = O(g)$
      \item $f = O(g)$
      \item $f = \Theta(g)$
      \item $f = \Omega(g)$
      \item $f = \Omega(g)$
      \item $f = O(g)$
      
    \end{multicols}
  \end{enumerate}
\end{solution}

\begin{exern}{2}
  Book 0.2
\end{exern}
\begin{solution}
    \hfill
    \begin{enumerate}
        \item If $c$ is less than $1$, $g(n)$ becomes a geometric series with common ratio, $c$. This series is equivalent to $\frac{1-c^n}{1-c}$ which is bounded between $0$ and $\frac{1}{1-c}$. Therefore, it is $\Theta(1)$.
        \item If $c$ is equal to $1$, then all terms of the series will also be $1$. Because there are $n$ terms, the series converges to $n$. Therefore, it is $\Theta(n)$.
        \item We only care about the  dominant term in the series, we will drop terms with lower power and focus on the $c^n$. Therefore, it is $\Theta(c^n)$
    \end{enumerate}
\end{solution} 

\begin{exern}{3}
  Book 1.11: Is $4^{1536} - 9^{4824}$ divisible by $35$?
\end{exern}
\begin{solution} 
  We can show it is divisible by taking each term modulo $35$.
  \[4^{6k} \equiv 1 \Mod{35} \forall k \in \mathbb{Z}\]
  \[4^{1536} \equiv 4^{6\cdot256} \equiv 1 \Mod{35}\]
  Similarly,
  \[9^{6k} \equiv 1 \Mod{35}\]
  \[9^{4824} \equiv 9^{6\cdot804} \equiv 1 \Mod{35} \]
  Therefore, we can simplify the original statement
  \[4^{1536} - 9^{4824} \equiv 1 - 1 \equiv 0 \Mod{35}\]
  Therefore $4^{1536} - 9^{4824}$ is divisible by 35.
\end{solution}

\begin{exern}{4}
    Book 1.12: What is $2^{2^{2023}}\Mod 3$?
\end{exern}
\begin{solution}
    Notice that \[2 \equiv -1 \Mod 3\]
    \[2^k \equiv {(-1)}^k \Mod 3\]
    If $k$ is even, we can simplify this to
    \[2^k \equiv 1 \Mod 3\]
    Because $2^{2023}$ is even, we can set $k$ equal to $2^{2023}$
    \[2^{2^{2023}} \equiv 1 \Mod 3\]
    And we are done.
\end{solution}

\pagebreak

\begin{exern}{5}
    Book 1.14: Suppose you want to compute the $n^{th}$ Fibonacci number modulo $5$. Describe the most efficient way in which you can do this.
\end{exern}
\begin{solution}
    We begin by solving the Fibonacci recurrence relation. This yields the following formula:
    \[F_n = \frac1{\sqrt5} {\left(\frac{1 + \sqrt 5}2\right)}^n + \frac1{\sqrt5}{\left(\frac{1 + \sqrt5}2\right)}^n\]
    This formula requires us to only calculate exponentiation for the $n^{th}$ Fibonacci number. Because exponentiation can be done in logarithmic time and exponentiation being the most time-consuming action here, the algorithm would be $O(\log(n))$. Note that exponentiation only runs in logarithmic time for  small n, this means that for large n, the complexity changes.
\end{solution}

\begin{exern}{6}
    Grad student A has designed an algorithm whose running time is ${\log(n)}^{\log(n)}$. Grad student B has designed an algorithm whose running time is $\frac n{\log(n)}$. Which student has the better algorithm as $n$ goes to $\infty$?
\end{exern}
\begin{solution}
    Grad student B has a vastly better algorithm as $n \to \infty$. B's algorithm is faster than linear time and A's is worse than exponential time.
\end{solution}

\begin{exern}{7}
    Book 1.17
\end{exern}
\begin{solution}
    The iterative approach requires $y-1$ multiplications therefore, in terms of $y$ and $n$, this algorithm has a running time of $n^2(y-1)$. The recursive approach would have a running time of $n^2\log(y-1)$ because there are $\log(y-1)$ multiplications. Overall, the recursive approach is more efficient as it runs in logarithmic time.  
\end{solution}

\pagebreak

\begin{exern}{8}
    Book 1.20
\end{exern}
\begin{solution}
    \hfill
    \begin{enumerate}
        \item $4$ because $4 \cdot 20 = 80 \equiv 1 \Mod {79}$ 
        \item $21$ because $21 \cdot 3 = 63 \equiv 1 \Mod {62}$
        \item The inverse does not exist because $21$ and $91$ are not coprime as they are both divisible by $7$.
        \item $14$ because $14 \cdot 5 = 70 \equiv 1 \Mod {23}$
    \end{enumerate}
\end{solution}



\end{document}
